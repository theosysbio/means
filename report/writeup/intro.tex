\section{Introduction} \label{intro}
\subsection{Moment Closure Methods}


%why studing systems
Most biological systems, such as cells, organisms, populations and ecosystems are intrinsecaly comlex and non-linear.
For this reason, some aspect of these systems are extremely difficult to understand and predict, both qualitatively and quantitatively.
Using explicit models to describe them provide an abstract and extensible framework that allows to infer starte of a system from experimental data. 
In addition, models permit to meake testable predictions on the behaviour of the system when parameters are modified.
In the last few decades, using mathematical representations of biological interactions has been an increasingly promissing and wide aspect of research in biology.
Many areas of biology, such as ecology, population biology or and biochemestry, use kinetic modeling to describe and understand temporal dynamics of their repective systems.

%Deterministic modeling (pro and limits)
Deterministic modeleing of dynamic systems involves 

This approach has been extremely usful for describing many systems, but it faces severe limitations when modeleing small discrete quantities.
For inscance, macromolecules in a cell can be present in a very small amount (less than 100). 
In this situation, the assumption of deterministic modelling will not be met. 
This has been shown to result in quantitative inacuracy, or, in the worst cases, qualitatively eroneous predictions.


%Stochastic modeling (pro, limits)
In order to overcome limitations of deterministic modeling, stochastic modeling has been advances as a solution.
It relies on \gls{cme}, which is a set of differential (or difference) equations providing an \emph{exact} description of the system.
Exept for very simple systems, the \gls{cme} cannot be solved analytically. 
However, it is possible to simmulate single realisations of the chemical master equation using \gls{gssa}.
If enough (generaly several thouthands), simulations are perfomed, very accurate estimations of the system can be obtained.
Despide extensive effort on increasing the efficiency of such simulations, either by describing new algorithms, or by improving implementations,
they remain too slow for certain problems. 
In particular, when trying to infer biological parameters from experimental data, according to an explicit model,
it is necessary to perform many simulations. 
In extreme cases, such as for \gls{abc}, a very large number of simulations are necessary in order to obtain an accurate posterior distributions for parameters.
For prameter inference, stochatsic simulations are unfortunately  criticaly slow and cannot be used, even for relatively simple systems.


In order to overcome performance limitations of stochatsic simulations whist providing accurate enough results, approximations have been used.
\Gls{lna}, for instance, approximates the 
\gls{cme}
by taking in acount only the means (first order raw moments), and the variances and covrariances (second order central moments).
This method has been shown to be valid when the amount of each species is large and for systems with first order reactions (such as  a -> b, but not a + c  -> b).
Another appraoch has been to expand the \gls{cme} by expressing central moment in terms of second or third order central moment.
Recently, this concept has been generalised to expand the \gls{cme} up to any arbitrary moment order.
\cite{ale_general_2013}
This method, \gls{mea}, has shown very promissing results, but has not yet been investigated in detail. 
In addition, no comprehensive public implementation is avalable for the community. 



%%%%%%%%%%%%%%%%%%%%%%%%%%%%%%%%%%%%%%%%%%%%%%%%%%%%%%%%%%%%%%%%%%%%%%%%%%%
%closure

%what is moment closure
To model the changes in concentrations using \gls{mea}, higher order moment are required to express the lower order moments, i.e. the $ith$ moment depends on the $(i+1)th$ moment. 
This makes the expressions for moment infinitely long, and the \glspl{ode} impossible to solve. 
Therefore, a moment closure method is necessary to approximate the higher order moments either by assuming the higher order moments are zeros, or by assuming a parametric distribution of the population. 

%non-parametric vs parametric distribution
Assuming zeros of all higher order moments reduces the computation effort, as the higher order moments are simply neglected.
However, this method does not exploit any parametric distribution for the population, and would result in inaccurate simulations if the type of distribution was known. 
In contrast, several parametric moment closure methods are available, where they estimate higher order moments using an assumed distribution type.
This allows researchers to use the prior knowledge to model the system more precisely. 

In \means, we have deployed both non-parametric (or scalar) and three parametric closure methods - Normal, Log-normal and Gamma closure.
For each method, the data can be either univariate or multivariate. 
Univariate is used when only one species is modelled in a system, or when multiple species can be regarded as one species, for instance, monomers in dimerisation reactions. 

Each of the parametric method is based on a theorem.
Normal closure uses Isserlis\rq theorem (or Wick\rq s theorem) to compute higher order moments in terms of the covariance matrix.\todo{cite On a formula for the product-moment coefficient of any order of a normal frequency distribution in any number of variables} 
The theorem assumes that the population distribution for all the species in the system is a zero mean multivariate normal random vector. 
It approximates the even order moments as the sum of products of all the possible pairs of partitions, whereas the odd order moments are defined to be zeros. 
For example, the fourth order moments $\mathrm{E}[x_1x_2x_3x_4] = \mathrm{E}[x_1x_2]\ \mathrm{E}[x_3x_4] + \mathrm{E}[x_1x_3]\ \mathrm{E}[x_2x_4]+\ \mathrm{E}[x_1x_4]\ \mathrm{E}[x_2x_3]$. 
In univariate cases, the therem works by simply setting the covariances and to zeros. 

Log-normal closure was first documented by Crow. \todo{cite the book Theory and Applications} 
Similarly, Log-normal closure also estimates higher order moments in terms of Log-covariance matrix, but the odd order moments are estimated based on the algorithm described in  Singh \& Hespanha, 2006, instead of being set to zeros. \todo{cite Lognormal Moment Closures for Biochemical Reactions} The advantage of using Log-normal closure, compared with Normal closure, is that its skewness and non-negative properties better describes populations in reality.\todo{cite Novel moment closure approximations in stochastic epidemics}

Current two definitions of Gamma distribution result in different types of Gamma moment closure methods.\todo{cite On a multivariate-gamma and On a multivariate gamma distribution} Both definitions of multivariate Gamma distribution use the parameters in the distribution to express the first and second order moments, and any higher order moments can be derived from multinomial formula. \todo {cite Eszter unpublished} One of the limitations of Gamma closure is incorrect modelling for systems with negatively correlated species. 
Moreover, prior knowledge about the population distribution is critical in choosing the parameter values for Gamma distribution.  

%%%%%%%%%%%%%%%%%%%%%%%%%%%%%%%%%%%%%%%%%%%%%%%%%%%%%%%%%%%%%%%%%%%%%%%%%%%%%%%%%

% bottleneck
The aim of our work was to provide a comprehensive implementation of \gls{mea} with parametric moment closure. 
We focused on maintanability, usability and performance. 
We provide \means, a new \py{} package for Moment Expansion Approximation, Inference and Simulations.
Implementing \gls{mea}, but also \gls{gssa} and \gls{lna}. 
In the report herein, we explain which decisions drove the design of the packge. 
Then, we explain how we have structured the code and improved maitanability by writing unittests.
Some example of intended use are presented to illustarte usability.
We then demonstate how we manages to dramaticaly improved performance during development.
Importantly, we further have used our package to critically assess \gls{mea}.
Finally, Our package also comes with an exhaustive documentation as well as a large set of tutorials, both printed in the appendix of this document.


blabe bla \cite{ale_general_2013}

