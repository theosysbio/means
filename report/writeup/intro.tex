\section{Introduction} \label{intro}


%why studing systems
Most biological systems, such as cells, organisms, populations and ecosystems are intrinsecaly comlex and non-linear.
For this reason, some aspect of these systems are extremely difficult to understand and predict, both qualitatively and quantitatively.
Using explicit models to describe them provide an abstract and extensible framework that allows to infer starte of a system from experimental data. 
In addition, models permit to meake testable predictions on the behaviour of the system when parameters are modified.
In the last few decades, using mathematical representations of biological interactions has been an increasingly promissing and wide aspect of research in biology.
Many areas of biology, such as ecology, population biology or and biochemestry, use kinetic modeling to describe and understand temporal dynamics of their repective systems.

%Deterministic modeling (pro and limits)
Deterministic modeleing of dynamic systems involves 

This approach has been extremely usful for describing many systems, but it faces severe limitations when modeleing small discrete quantities.
For inscance, macromolecules in a cell can be present in a very small amount (less than 100). 
In this situation, the assumption of deterministic modelling will not be met. 
This has been shown to result in quantitative inacuracy, or, in the worst cases, qualitatively eroneous predictions.


%Stochastic modeling (pro, limits)
In order to overcome limitations of deterministic modeling, stochastic modeling has been advances as a solution.
It relies on \gls{cme}, which is a set of differential (or difference) equations providing an \emph{exact} description of the system.
Exept for very simple systems, the \gls{cme} cannot be solved analytically. 
However, it is possible to simmulate single realisations of the chemical master equation using \gls{gssa}.
If enough (generaly several thouthands), simulations are perfomed, very accurate estimations of the system can be obtained.
Despide extensive effort on increasing the efficiency of such simulations, either by describing new algorithms, or by improving implementations,
they remain too slow for certain problems. 
In particular, when trying to infer biological parameters from experimental data, according to an explicit model,
it is necessary to perform many simulations. 
In extreme cases, such as for \gls{abc}, a very large number of simulations are necessary in order to obtain an accurate posterior distributions for parameters.
For prameter inference, stochatsic simulations are unfortunately  criticaly slow and cannot be used, even for relatively simple systems.


In order to overcome performance limitations of stochatsic simulations whist providing accurate enough results, approximations have been used.
\Gls{lna}, for instance, approximates the 
\gls{cme}
by taking in acount only the means (first order raw moments), and the variances and covrariances (second order central moments).
This method has been shown to be valid when the amount of each species is large and for systems with first order reactions (such as  a -> b, but not a + c  -> b).
Another appraoch has been to expand the \gls{cme} by expressing central moment in terms of second or third order central moment.
Recently, this concept has been generalised to expand the \gls{cme} up to any arbitrary moment order.
\cite{ale_general_2013}
This method, \gls{mea}, has shown very promissing results, but has not yet been investigated in detail. 
In addition, no comprehensive public implementation is avalable for the community. 



%%%
%closure
%%%

% bottleneck
The aim of our work was to provide a comprehensive implementation of \gls{mea} with parametric moment closure. 
We focussed on maintanability, usability and performance. 
We provide \means, a new \py{} package for Moment Expansion Approximation, iNference and Simulations.
Implementing \gls{mea}, but also \gls{gssa} and \gls{lna}. 
In the report herein, we explain which decisions drove the design of the packge. 
Then, we explain how we have structured the code and improved maitanability by writing unittests.
Some example of intended use are presented to illustarte usability.
We then demonstate how we manages to dramaticaly improved performance during development.
Importantly, we further have used our package to critically assess \gls{mea}.
Finally, Our package also comes with an exhaustive documentation as well as a large set of tutorials, both printed in the appendix of this document.
