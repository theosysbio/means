\section{Conclusion} \label{sec:conclus}

The aim of this project was to provide an efficent \py{} implementation of \acrlong{MEA}.

We made the decision to develope a python package as opose to a command line application.
We used the code from the previous years student, which we eventualy completely restructured and recoded, as a base to build our package.
We have used state-of-the-art software develpment standards, such as continuous integration and unittests in order to manage our project.
This resulted in an easy to usem documented and mainable, package.

We implemented moment expansion closure methods.
we implemented support for the SBML stadard,
We implemented more common algorithms such as \gls{gssa}, and \gls{lna}.

We provide tutorial

A considerable effort was made for improving performance of \gls{mea}.


We beleive \means{} will be a valuable assess for reaserchers in our field. In addition, 
 


\begin{verbatim}
* we restructured an existing program into a fully fledged package
* We used modern development tools (ipython nb, jenkins, \ldots)
-> documented, maintainable, usable
* we added support for more conventional tools (SSA, reading sbml)
* We considerably improved performance

* We implemented interesting features (Closure, Sensitivity,
support for multiple solvers)

* Our package allows us to investigate MEA.
* We discovered some limitations:
-
-
-

* There is still a lot of work to do before MEA can can become mainstream

* (Un)Availability of our package

\end{verbatim}

\todo[inline]{Once we are all good, we need to zip the task-output data and make it share it somewhere as well as whole report depends on it}
