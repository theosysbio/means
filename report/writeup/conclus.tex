\section{Conclusion} \label{sec:conclus}

The goal of this project was to develop an efficient \py{} implementation of \acrlong{mea}.
We aimed at providing a tool that would be powerful whilst simple to maintain and extend.
After thoroughly considering the requirements and potential impact, we decided to write a \py{} package as opposed to a command line application.

Our work built on top of the available \mat{} code as well as last year's MSc students' project, which we eventually completely restructured.
We used state-of-the-art software development standards, such as version control, continuous integration and unittests to manage our project and speed-up development.
In order to improve usability and maintainability, we provide an exhaustive series of tutorials and a detailed package documentation.

In addition, we implemented moment expansion closure with three parametric distributions (in their univariate and multivariate form).
In order to maximise impact for adoption of \means, we also implemented support for the SBML standard.
For completeness, and to allow comparison between different algorithms, \gls{gssa} and \gls{lna} were also implemented.
A considerable effort was made to improve the performance of \gls{mea},  in both symbolic computations and numerical evaluation.
We also provide a comprehensive tutorial showing detailed steps of executing each function of the package.

These improvements allowed us to investigate the effect of maximal moment order and closure distribution on \gls{mea} accuracy.
In contrast with the original publication, we find that the approximation does not necessarily improve with maximal moment order.
In addition, we discovered that according to the region of the parameter space, the best approximation (maximal order and closure distribution) could be very different.
We also realised that for large regions of the parameter space, all numerical solvers failed to simulate trajectories.
Finally, we pointed out that parameter inference based on \gls{mea} approximation cannot produce the expected parameter values, even though the inferred species trajectories can be a perfect match to the ``experimental'' data. Using higher order for \gls{mea} may improve the accuracy of parameter inference slightly, however, the true parameter values cannot be recovered even using maximal order of 6.
Reporting these unforeseen limitations will hopefully help researchers to improve simulation and inference using \gls{mea}.

Altogether, our implementation should facilitate further research and improve adoption of \acrlong{mea} by the community.
We believe \means{} could be a valuable assess for researchers to model non-linear kinetic system.
 
%We intend to release \means as open source software available 


\begin{verbatim}
* (Un)Availability of our package
\end{verbatim}

\todo[inline]{Once we are all good, we need to zip the task-output data and make it share it somewhere as well as whole report depends on it}
