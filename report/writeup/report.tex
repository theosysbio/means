\documentclass[11pt,a4paper]{article}
\setcounter{secnumdepth}{6}

\usepackage{standalone}
\usepackage{graphicx}
\usepackage[font=footnotesize]{caption}
\usepackage[noadjust]{cite}
\usepackage[toc,page]{appendix}
\usepackage{hyperref}
\usepackage{fullpage}
\usepackage{amsmath}
\usepackage{sidecap}
\usepackage{caption}
\usepackage{subcaption}
\usepackage{minted}
\usepackage{framed}
\usepackage[colorinlistoftodos]{todonotes}
\usepackage{csquotes}
\usepackage{parskip} % Spaces between paragraphs

\usepackage{pgffor} % foreach loops!

\usepackage[acronym]{glossaries}


\input{ipynbsupportdefinitions.tex}

\makeglossaries
\renewcommand*\abstractname{Summary}

% OUR COMMANDS
%~ 
%~ \newenvironment{pythoncode}%
%~ {\begin{framed}
%~ \begin{minted}{python}}%
%~ {\end{minted}
%~ \end{framed}}%

% http://www.tex.ac.uk/cgi-bin/texfaq2html?label=altabcr
\setcounter{MaxMatrixCols}{50}

% package name:
\newcommand{\means}{\texttt{MEANS}}
\newcommand{\pft}{\textit{p53}}
\newcommand{\py}{\texttt{python}}
\newcommand{\sympy}{\texttt{sympy}}
\newcommand{\plt}{\texttt{matplotlib}}
\newcommand{\mat}{\texttt{MATLAB}}
\newcommand{\eg}{\emph{e.g.}}
\newcommand{\ie}{\emph{i.e.}}

\newcommand{\sauliustodo}[2][]{\todo[color=cyan, #1]{\textbf{SL:} #2}}
\newcommand{\sisitodo}[2][]{\todo[color=yellow, #1]{\textbf{SF:} #2}}
\newcommand{\quentintodo}[2][]{\todo[color=red, #1]{\textbf{QG:} #2}}
\newcommand{\citationneeded}[2][]{\todo[color=brown, fancyline, #1]{\textbf{Citation Needed:} #2}}
\newcommand{\contrib}{\emph}
\begin{document}

\listoftodos
\newpage

\title{MEANS: a new python package for Moment Expansion Approximation, Inference and Simulation}
\author{Sisi Fan, Quentin Geissmann, Saulius Lukauskas.\\
\\	
\\
\\
\\
Supervised by Ann Babtie, Paul Kirk, Eszter Lakatos and Michael Stumpf\\
\\
\\
Theoretical Systems Biology Group,\\
Imperial College London
}
\date{\today}

\clearpage\maketitle
\thispagestyle{empty}
\newpage{}

\pagenumbering{roman}

\begin{abstract}
We present \means, a python package that provides stochastic modelling based on Moment Expansion Approximation (MEA), Linear Noise Approximation and Gillespie Stochastic Simulation Algorithm in an efficient and user-friendly way. Specifically for MEA, we have deployed scalar and three parametric moment closure methods in both univariate and multivariate forms. We also significantly improved MEA performance, which was a limiting factor in stochastic modelling.  In addition, the package also supports a collection of solvers and distance-based inference. To enhance usability and maintainability, we implemented modular design in the package with detailed documentation, as well as a series of interactive tutorials in IPython notebook format. In addition, our package supports SBML (Systems Biology Markup Language) standard for studying available biological systems. Apart from package development, we have conducted further analysis on MEA, and found higher using maximal moment order did not necessarily improve MEA accuracy. Moreover, we discovered that large parameter space could influence MEA accuracy and cause solver failure, which then might have led to incorrect inference based on MEA approximation. These limitations may provide insight into improvements of simulation and inference using MEA. We believe MEANS will a valuable package to study MEA and facilitate researchers to model non-linear kinetic systems. 
\end{abstract}

\tableofcontents

% our acronyms
\newacronym{ode}{ODE}{Ordinary Differential Equation}
\newacronym{mea}{MEA}{Moment Expansion Approximation}
\newacronym{lna}{LNA}{Linear Noise Approximation}
\newacronym{ssa}{SSA}{Gillespie Stochastic Simulation Algorithm}
\newacronym{cme}{CME}{Chemical Master Equation}
\newacronym{abc}{ABC}{Approximate Bayesian Computation}
\newacronym{sbml}{SBML}{Systems Biology Markup Language}
\newacronym{pypi}{PyPI}{the Python Package Index}

\newglossaryentry{maxord}
{
  name=max order,
  description={Maximal moment order. Max order always (regardless to the closure method) refers to the highest order of 
  moments present in an ODE system resulting from Moment Expansion Approximation. In other words, moment expansion is closed at Max order $\mathbf{+1}$
   },
  sort=max order
}
\gls{maxord}
\newpage{}
\printglossaries

\newpage{}
\pagenumbering{arabic}
\section{Introduction} \label{intro}
\subsection{Moment Closure Methods}
To model the changes in concentrations using \gls{mea}, higher order moment are required to express the lower order moments, i.e. the $ith$ moment depends on the $(i+1)th$ moment. This makes the expressions for moment infinitely long, and the \glspl{ode} impossible to solve. Therefore, a closure method is necessary to approximate the expressions for higher order moments either by assuming the higher order moments are zeros, or by assuming a parametric distribution of the population. Moment closure is particularly useful for stochastic population models, as it leads to a set of \emph{"closed"} solvable differential equations to model the system. \todo{cite Moment-closure approximations for mass-action models}

Assuming zeros of all higher order moments reduces the computation effort, as the higher order moments are simply neglected. However, this method does exploit any parametric distribution for the population, resulting in inaccurate simulation if the type of distribution was known. 

In contrast, several parametric moment closure methods are available, where they estimate higher order moments using an assumed distribution type. This allows researchers to use the known distribution type to model the system more precisely. 

In \means, we have deployed both non-parametric and three parametric closure methods - Normal, Log-normal and Gamma closure. For each method, the data can be either univariate or multivariate. Univariate is used when only one species is modelled in a system, or when multiple species can be regarded as one species, for instance, in dimerisation reactions. 

Each of the parametric method is based on a therem. Normal closure uses Isserlis\rq , theorem or Wick\rq s theorem to compute higher order moments in terms of covariance matrix.\todo{cite On a formula for the product-moment coefficient of any order of a normal frequency distribution in any number of variables} The theorem assumes that the population distribution for all the species in the system is a zero mean multivariate normal random vector. It approximates the even order moments as the sum of products of all the possible pairs of partitions, whereas the odd order moments are defined to be zeros. For example, the fourth order moments $\mathrm{E}[x_1x_2x_3x_4] = \mathrm{E}[x_1x_2]\ \mathrm{E}[x_3x_4] + \mathrm{E}[x_1x_3]\ \mathrm{E}[x_2x_4]+\ \mathrm{E}[x_1x_4]\ \mathrm{E}[x_2x_3]$. In the univariate case, the therem works by simply setting the covariances and to zeros. 

Log-normal closure was first documented by Crow. \todo{cite the book Theory and Applications} Similarly, Log-normal closure also estimates higher order moments in terms of log-covariance matrix, but the odd order moments are estimated based on the algorithm described in  Singh \& Hespanha, 2006. \todo{cite Lognormal Moment Closures for Biochemical Reactions} The advantage of using Log-normal closure, compared with Normal closure, is that its skewness and non-negative properties better describe populations in reality.\todo{cite Novel moment closure approximations in stochastic epidemics}

Current definitions of Gamma distribution result in different types of Gamma moment closure methods.\todo{cite On a multivariate-gamma and On a multivariate gamma distribution} Both definitions of multivariate Gamma distribution use the parameters  to express the first and second order moments, and any higher order moments can be derived from multinomial formula. \todo {cite Eszter unpublished} One of the limitations of Gamma closure is incorrect modelling for systems with negatively correlated species. Moreover, prior knowledge about the population distribution is critical in choosing the parameter values for Gamma distribution. 

Using \means, we intend to further explore the quality of these closure methods using several bio-models, and the results are analysed in Results and Discussion.
 
blabe bla \cite{ale_general_2013}

\newpage
\section{Approach to Software Design}
\label{sec:design_considerations}

Our software development has been heavily inspired by the Agile methodology\cite{_manifesto_????}.
Agile methodology has been established as an alternative to the traditional waterfall method which views requirements gathering, software design and implementation as a linear process.
In contrast to this traditional view, agile methodology centres around iterative approach to software development, and aims to minimise the cost caused by problems along the way.

The key principles of this methodology have been neatly summarised by Kelly Waters\cite{_what_????}:

\blockquote{
    [The] characteristics that are common to all agile methods, and the things that I think make agile fundamentally different to a more traditional waterfall approach to software development [\ldots] are:
    
    \begin{enumerate}
        \item Active user involvement is imperative
        \item The team must be empowered to make decisions
        \item Requirements evolve but the timescale is fixed
        \item Capture requirements at a high level; lightweight \& visual
        \item Develop small, incremental releases and iterate
        \item Focus on frequent delivery of products
        \item Complete each feature before moving on to the next
        \item Apply the 80/20 rule
        \item Testing is integrated throughout the project lifecycle -– test early and often
        \item A collaborative \& cooperative approach between all stakeholders is essential
    \end{enumerate}
}

While these may sound very vague requirements, they capture the recommendations based on years of trial-and-error on software development in the industry, and generally act as good guidelines to follow for projects with short deadlines.

\subsection{Requirements Gathering}
One of the key parts of any project is the requirements gathering process -- no amount of skill or hard work will help if the work is misdirected.
Naturally, we took great interest in getting these requirements right.

Capturing requirements at ``\emph{at a high level}" was important for this project since the temptation to discuss about low-level implementation before gathering requirements was great.
These discussions are often constraining the requirements with the implementation details, while ideally we wanted the requirements to influence the implementation decisions, and not the other way around.

The easiest way to keep the requirements in a high level was to limit the discussion to a set of scenarios of usage.
These scenarios are commonly referred to \emph{user stories} and usually follow the template: ``As (some user type, \eg{} \emph{a scientist)} I would like to (some action, \eg{} \emph{perform these calculations}) because (optionally, the reason for this use case)".
The first part of this relatively free-form gives a view of the potential user base of the software.
For instance researchers and undergraduate students may have different requierements.
The second part, provides a set of important use cases for the problem.
Finally, the third part assesses the relative importance of each of the use cases.

We were able to write down four representative user stories, listed below:
\begin{enumerate}

    \item \emph{As a researcher studying biological systems, I would like to perform a variety of simulations, with different parameter sets, and compare the results of these simulations.}

    \item \emph{As a researcher studying the approximation method, I would like to be able to compare the changes to the approximation quality, when the approximation parameters (\ie{} the closure method) are changed.}

    \item \emph{As a researcher studying biological systems, I would like to be able to infer the set of parameters of the model given the data, but this idea is still not very popular in my field and I am not sure if it is feasible.}

    \item \emph{As a systems biologist, I have to deal with multiple data formats on case by case basis, and would like to support them all or have a simple format that could be easily writeable, because I am computer-savvy enough to write a conversion script.}
    \todo{QG: I struggle with this last story.}
\end{enumerate}

The pattern emerging from these use cases was that our project software should not be designed to act as a standalone piece of software, \todo{QG: second part of sentence ambiguous}and merely be an intermediate step in some larger data analysis step.
For instance, the \emph{comparison} process in the first two user stories could not be clarified in more detail, because the techniques employed in that step are chosen in a very \emph{ad-hoc} fashion.

Similarly, the fourth user story indicates that the origin of the input data and its format are undefined.
Therefore the software needed be general enough to support multiple input formats, or have a simple enough format allowing researchers to convert their data to.
This was not unsurprising, given the exploratory nature of the method,\todo{QG: not clear} of course, yet not less important to achieve.

This observation encouraged us to move away from the traditional command-line application model for this software\todo{cite last year report}.
In our opinion, flexibility of command-line applications comes at the expense of simplicity -- we have all seen really flexible applications that have a list of parameters that is too long for a newcomer to comprehend.
Similarly, command-line applications create an artificial gap between the data processing and data analysis step, \ie{} writing and reading file output, which can be reduced if the application was returning the data in the format that can readily be used to investigate the data.

\todo{"Instead, we ... " We need to bi rigth to the point : we made a python package!!, it is called MEANS! :D}
We believe that the best way we can achieve both flexibility and minimise the gap between the data processing and analysis with the help of package-like structure and interactive python environments, such as IPython notebook\todo{Cite ipython notebook}.
We therefore have spent our development efforts on designing an intuitive modular system for the package, and ensuring integration with this environment.
This structure is further described in \autoref{sec:package}.


\subsection{Package Structure}
\label{sec:package}

In order to make the code easier to maintain and manage, we have divided \means{} package into nine different smaller submodules, each responsible for a different part of the system.
We have paid particular attention in making all the necessary features importable using the pythonic shorthand \verb"from means import *", 
as to ensure it is simple enough to use \means{} in interactive environments where it might be uncomfortable to deal with modularity explicitly.

The nine submodules in \means{} ands their respective role are listed below:
\begin{enumerate}
    \item \verb"means.core" -- the core functionality, such as the definitions of key classes
    \item \verb"means.approximation" -- implementations of the \gls{lna} and \gls{mea} approximation methods.
    
    \item \verb"means.simulation" -- implementation of various \gls{ode} solving algorithms.
    \item \verb"means.inference" -- implementation of parameter inference methods
    \item \verb"means.io" -- implementation of input and output routines
    \item \verb"means.util" -- utility functions used across the modules, internal to the package
    \item \verb"means.tests" -- various tests for the package functionality, hidden from the user by default
    \item \verb"means.examples" -- definitions of example models that are used throughout tutorials, needs to explicitly be imported by the user.
    \item \verb"means.pipes" -- optional module providing wrappers for pipeline support
\end{enumerate}

Each submodule is described in detail below.

\subsubsection{{\tt means.core} -- core classes used within all other packages}
% If you are editing this, please also edit src/means/core/__init__.py docstring
% to keep the two consistent
This package stores the common classes that are used within all of the other means sub-packages.

The module exposes the descriptor classes, such as \verb"Descriptor",
\verb"VarianceTerm", \verb"Moment" and
\verb"ODETermBase" that are used to describe the types of trajectories generated,
as well as certain terms in the ODE equations.

Similarly, both the \verb"StochasticProblem" and
\verb"ODEProblem" classes that are used in stochastic and deterministic simulations respectively are exposed by this module.

Finally, the \verb"Model" class, which provides a standard interface to describe a biological model, and can be thought to be the center of the whole package, is also implemented here.

\subsubsection{{\tt means.approximation} -- approximation methods for biological models}

The approximation submodule implements \gls{lna} and \gls{mea} methods for biological systems encoded by the previously mentioned \verb"Model" objects (see \autoref{sec:making-a-model} for a detailed description on how this encoding is done).

The approximation package is itself split into two sub-packages, one for each of the approximation methods mentioned.
This keeps specific routines in separate modules, thus avoiding \gls{lna} and \gls{mea} implementations namespace polutiong each other.
This also provides a clear path for future developers to implement new approximation methods -- simply creating another sub-package.

Despite being in different packages, both approximation methods inherit from the same base class, intuitively named \verb"ApproximationBaseClass".
This base class defines the common interface all approximation methods; namely, the dependancy on \verb"Model" and public \verb"run()" method that implements the mathematics and returns a \verb"ODEProblem" object that can be simulated.
This common interface is required for any pipeline using the approximation methods to be method-agnostic.
In other words, we want the pipeline to be able to perform the approximation in exactly the same way, regardless of whether \gls{lna} or \gls{mea} was used. 
It is easy to see how this design pattern increases the code maintainability and compatibility as well: not only the future maintainer will have a clear template to follow for implementing an approximation method, but it will also be immediately compatible with all the packages using the approximations.

The advantages described above apply mostly for software using all or part of the \means{} package to perform designated tasks\todo{QG: not v clear}.
The necessity of creating approximation objects and running their \verb"run()" function in order obtain the \gls{ode}s soon becomes a burden in the interactive environments.
In order to overcome this, we also provide shorthand functions \verb"mea_approximation" and \verb"lna_approximation", the use of which is demonstrated in \autoref{sec:approximation-example}.

\subsubsection{{\tt means.simulation} -- stochastic and deterministic simulation routines}

The next submodule in the list provides the routines for performing deterministic and stochastic simulations.
These simulations are performed via \verb"Simulation" and \verb"SSASimulation" classes defined in this sub-module.

Unlike the approximation package, the simulation routines could not share the same interface, as they have different inputs -- a stochastic simulation needs \verb"StochasticProblem" that provides the stoichiometry matrix as well as hazard functions to simulate stochastically, while the deterministic simulation routine requires an \verb"ODEProblem" object that is essentially a collection of ordinary differential equations.
Nevertheless, the classes still share the same interface, providing a \verb"simulate_system()" method which in turn generates a set of \verb"Trajectory" objects, to keep them consistent.

Also unlike the approximation submodule, we deliberately do not provide the convenience methods for performing simulations and force the user to create the \verb"Simulation" objects herself. 
Firstly, the convenience functions for simulations would have to have a lot of parameters, in order to be as flexible as the explicit creation of the simulator objects. 
This is a non-issue for the approximation objects, whose heavy-lifting function takes zero parameters, so all of the complexity is in the constructor parameters only.
Secondly, due to the way we perform numerical evaluations, subsequent uses of the same \verb"Simulation" object are much cheaper than the very first use of it which has to precompile the numeric routines. 
Shorthand function would not provide a way to exploit this caching of computation, as it would force to create a new simulation object, even if it could be reused, therefore we did not code that workflow.

Examples on how to use both classes are provided in \autoref{sec:example-simulation}.

\subsubsection{{\tt means.inference} -- parameter inference methods}
% Change src/means/inference/__init__.py if you are modifying this
This part of the package provides utilities for parameter inference.
Parameter inference will try to find the set of parameters which
produces trajectories with minimal distance to the observed trajectories.

Different distance functions are implemented (such as functions minimising the sum of squares error
or functions based on parametric likelihood), but it is also possible to use custom distance functions.

The package provides support for both inference from a single starting point (\verb"Inference")
or inference from random starting points (\verb"InferenceWithRestarts").

Some basic inference result plotting functionality is also provided by the package, see the documentation for
\verb"InferenceResult" class for more information on this. \todo{Link to documentation/provide it in examples section}

\subsubsection{{\tt means.io} -- reading and writing of \means-specific objects}
% change src/means/io/__init__.py if you are changing this
The module implements common methods to serialise and deserialise \means objects.
Namely the module provides functions \verb`means.io.dump` and  \verb`means.io.load` that would
serialise and deserialise the said objects into \todo{figure out how I want to cite this}YAML format.
These serialised representations can be written to or read from files with the help of
\verb`means.io.to_file` and \verb`means.io.from_file` functions.
% this is not in __init__.py, but is required for report
The YAML format has been chosen due to it's ability to balance human and machine readability. 
Some examples of the YAML serialised objects can be seen in the \todo{write IO tutorial and link here}.

%this is in __init__.py again
For the user's convenience, the said methods are also attached to all serialisable objects,
e.g. \verb`means.core.Model.from_file()` method would allow the user to read a \verb`means.core.Model` object from file directly.

%also in __init__.py
We do not provide any convenience functions for binary serialisation of the object, because \verb`pickle` package,
which is in the default distribution of Python, has no problems of performing these tasks on \means objects. 
We recommend using \verb`pickle`, rather than \verb`means.io` whenever fast serialisation is preferred to human
readability.

% still in __init__.py
Finally, this module also provides support for the input from SBML files.
If the \verb`libsbml` is installed in the user's system and has the appropriate python bindings, the function \verb`means.io.read_sbml` can be used to parse the files in SBML format
to \verb`means.core.Model` objects. See example usage in \todo{Link to some tutorial}.

\subsubsection{{\tt means.util} -- utility functions used within the package}
% change src/means/util/__init__.py if you're changing this
This package contains helper functions that did not fit into any other packages.
These functions include helper functions for common operations when dealing with \verb`sympy`,
as well as functions that help with memoisation of CPU intensive function results.

These functions are designed to be package-specific.
The users of the software are generally discouraged to use any of these functions.

\subsubsection{{\tt means.tests} -- ensuring the working implementation}
The tests package contains, rather unsurprisingly, the tests for our implementation.
Our approach to testing is described in \autoref{sec:testing}

\subsubsection{{\tt means.examples} -- example models used in the documentation}
The examples package contain a set of toy models used in the documentation.
The models included with the package are:
\begin{itemize}
    \item a sample Michaelis-Menten kinetics model as in \cite{ale_general_2013};
    \item Dimerisation model, also in \cite{ale_general_2013};
    \item P53 model, described in \cite{ale_general_2013} and in \autoref{fig:p53} of this document;
    \item HES1 model, based on the description in \cite{ale_general_2013} \todo{Was it in ale?};
    \item A textbook example of Lotka-Volterra dynamics \todo{QG: cite the original LV book/paper}.
\end{itemize}

These example models are designed to help us to illustrate important concepts of the package in the documentation, but are otherwise independent from the package, and therefore need to be imported separately.
The user of the software is nevertheless encouraged to experiment with these toy models.

\subsubsection{{\tt means.pipes} -- convenience functions for data pipelines}
The \emph{pipes} submodel provides the convenience functions for pipeline tasks using \verb"luigi" package. \todo{Cite luigi somehwere, figure out how}.

The pipeline support is covered in detail in \autoref{sec:pipelines}.

\subsection{Testing}
\label{sec:testing}
\todo{Testing testing testing OY!}
\subsection{Pipelines}
\label{sec:pipelines}
\todo{Let's use this section to explain how we generated this report.}


\newpage{}
\section{Example of Usage} \label{examples}
In this section, we desmonstrate a simple usage of \means. 
A collection of detailed tutorials is available in Appendix~\ref{tutorials}.

A common example if the tummor suppresior \pft system. 
Whist relatively uncomplicated, it shows an shows an interesting oscilatory behaviour and has been exhaustively studied \cite{geva-zatorsky_oscillations_2006, batchelor_ups_2009}.
The model can be formulated as a list of species, a list of reaction between species, and a stoichiometry matrix.
For this system, we will consider the species $$, $$, $$

\subsection{Making a Model}
The first







\newpage{}
\section{Results and Discussion} \label{results}



\subsection{Performance}\label{performance}
\sauliustodo{I think this should be moved somewhere earlier in the report as it justifies the pretext of our work}
The temporal dynamics of a molecular system can be described by the \acrfull{cme}\citationneeded{CME reference}.
\acrfull{gssa}\cite{gillespie_general_1976} simulates these stochastic dynamics directly returning a set of individual trajectories for a list of particles observed.
In order to obtain accurate estimates of the average dynamic within a population of cell (\ie{} the mean dynamics), it is however necessary to perform multiple (often more than $10^4$) simulations.
Despite recent efforts \cite{niemi_efficient_2011,dittamo_optimized_2009,komarov_accelerating_2012} to provide fast implementation of this algorithm, computation remains extremely expensive. \quentintodo{Do you know the Big-Oh complexity?} This time-complexity is a limiting factor in downstream analysis techniques,
for instance parameter inference that often requires repetition of these experiments for a large set of parameter values.

This particular limitation led to the development of approximations such as \acrfull{lna}\cite{komorowski_bayesian_2009} and \acrfull{mea}\cite{ale_general_2013} which model the mean behaviour directly, without evaluating the individual particle behaviours, and therefore can perform in a more realistic time.

Since the main driving force for development of these approximation algorithms is the potential reduction of the time taken to perform the  analysis, it is paramount to make the implementation as efficient as possible.

In this section, we explore the implementation factors that influence the runtime of the algorithm, and describe the optimisations done to increase the performance of it. In particular, we show that symbolic computations can be limiting the algorithm and explain the techniques used to optimised them. We also quantify the increase of performance  and show that it is several order of magnitudes better than the original \mat{} implementation.
In addition, we explore other limiting factors we have less control of, such as the choice of \gls{ode} solver, and discuss their potential implications to the analysis of biological systems.

\subsubsection{Optimising \acrlong{mea}}
\label{sec:optimising_mea}


\gls{mea} involves derivation of a system \gls{ode}s from a model.
This procedure\cite{ale_general_2013}, involves lengthy symbolic calculations.
Even for very simple models (\eg{} three species, five reactions), they cannot be realised manually.
The number equations in the generated \gls{ode} system, for a model with $s$ species and up to moments of order $o$ can be estimated by the following equation: 
\quentintodo[inline]{Does not make sense for 3 species, and 2nd-order moments (3+2-2 choose 3) - 1 = 0}
\begin{displaymath}
    \text{Number of equations} = {{s + o - 2} \choose {s}} - 1
\end{displaymath}
As a consequence, the complexity of the calculation is predicted to increase exponentially with the number of species in the system and the maximal order of moments. \quentintodo[inline]{Add an example numbers illustrating this: "For instance, for a 2-species system ...."}

%In order to perform symbolic computations, we have used \sympy{} \cite{sympy_development_team_sympy:_2014}; a \py{} implementation of
%the symbolic computation routines.

In order to increase the scalability of the method, we have identified significant bottlenecks in our procedures using \py{} profiling tools. We have then attempted to iteratively remove these bottlenecks one by one. Figure~\ref{fig:mea_speed} shows the cumulative effects of different optimisations.

\quentintodo[inline]{Explain why no original python performance line, reduce the detail in caption, make it more evident it is log-normal}


\begin{figure}[tbh]

\includegraphics[width=1.0\textwidth{}]{../figure_mea_speed/mea_speed.pdf}
\caption{\emph{Cumulative performance improvement of symbolic calculations resulting from optimisation}.
The processing time for computing log-normal closure on \pft{} model with different maximal moment orders were measured for original Matlab implementation (a) and different optimisations (b$-$f).
In a first place, the calls to \texttt{sympy.simplify()} where removed (b). 
Then, \texttt{sympy.xreplace()} was used instead of \texttt{sympy.substitute()} (c). 
Generating an $(n-s) \times (n_2-s + 1)$ matrix (d), as opposed to an $(n-s) \times (n-s + 1)$ one, also increase speed (see main text).
Implementing a simplified equation solver instead of using \texttt{sympy.solve()} also resulted in a significant speed-up (e). Finally, caching (memorisation) \texttt{sympy.diff()} allowed even better performance.
The time complexity appears exponential ($O(2^n)$, where $n$ is the maximal moments order) in every case, 
Interestingly, the slopes between, a ($0.95$) and c ($0.58$), and b ($0.62$) and c were significantly different ($p-value <10^{-15}$ and $p-value = 3 \times 10^{-4}$, respectively; t-test on the slopes of the linear regression). 
No significant difference was found between the slopes of the subsequent optimisations (c$-$f). 
However, the intercepts were significantly smaller between each consecutive optimisations after c) ($p-value < 10^{-6}$ for all; t-test on the intercepts of the linear regression).
Nine replicates were performed on the same CPU. For optimisation c$-$f, values corresponding to maximal order moments lower than two were removed because of the inherent inaccuracy in measuring very short durations}
\label{fig:mea_speed}
\end{figure}

\quentintodo[inline]{Simplify this, move out the specific details to your individual report, it is enough to give just the overview here.}

The first step involved removing the expression simplification heuristic.
In the original code\footnote{both from the publication and last year's MSc project}, the right-hand-side equations were simplified in order to produce shorter text file results.
However, this was slow and did not benefit subsequent simulations and inference.
For large expressions, simplification had also had an large memory footprint and was likely to fail.
This optimisation significantly improved the scalability of the method (see fig.~\ref{fig:mea_speed}, b).

The next bottleneck was the choice of substitution functions.
As a part of \gls{mea}, it is necessary to replace raw moment symbols by expressions depending on central moments.
Performing substitution can be done using the \texttt{substitute()} function from \sympy, but this is designed to substitute expressions by other expressions.
In most cases, we only had to substitute atomic symbols by expression.
For this purpose, the  \texttt{xreplace()} function was a much more appropriate alternative which resulted in a better scalability (see fig.~\ref{fig:mea_speed}, c).

In the original implementation, a matrix of central moment expression of size $(n-s) \times (n-s + 1)$ is directly generated when the default closure is applied.
However, when using a parametric closure, a matrix of size $(n_2-s) \times (n_2-s + 1)$, where $n_2={{s+o-2 \mathbf{+1}} \choose {s}} -1$, was generated.
The $n_2 - n$ rows corresponding to higher-order moments have then to be deleted.
In contrast, out implementation generates a $(n-s) \times (n_2-s + 1)$ matrix regardless of the closure method.
In addition to improve code readability, consistency and flexibility \footnote{see QG's individual report}, this improved overall performance (fig.~\ref{fig:mea_speed}, d) for cases where closure is applied while keeping the default closure computation fast.

Another simple way to improve computation time was to remove calls to the function \texttt{solve()} which was only used in straightforward cases (\eg{} solving: $a + 2b = c$ for $a$).
It was therefore much more efficient (fig.~\ref{fig:mea_speed}, e) to use simple arithmetic to find solution.

Finally, partial derivation of expression over several variables and order is extensively performed during the approximation.
Generally, these type of differentiations can be simplified several differentiation of first order:
\begin{equation}
\frac{\partial{} ^ 2 f(x,y)}{\partial x \partial y}  =
\frac{\partial{} \frac{\partial{} f(x,y)}{\partial x}}{\partial y} =
\frac{\partial{} \frac{\partial{} f(x,y)}{\partial{} y}}{\partial{} x}
\end{equation}
One advantage, is that, when needing to calculate two derivatives such as:  $\frac{\partial{} ^ 2 f(x,y)}{\partial{} x \partial{} y}$ and $\frac{\partial{} ^ 2 f(x,y)}{\partial{} x^2}$, 
one can precompute $\frac{\partial{} f(x,y)}{\partial{} x}$ and use it for both calculation.
In our implementation, we have use a procedure known as \emph{memorisation} which, briefly, permits to store the results of a function call in an associative array.
Then, the next time this function is called with the same arguments, it will return the stored results instead of recomputing it.
This also resulted in an overall performance improvement (fig.~\ref{fig:mea_speed}, f).

In conclusion, reorganising, profiling and rewriting the code resulted in incremental significant performance improvements of symbolic computations in \means{} compared to the original \mat{} code.
For instance, with the same \pft{} system and closure method, 
we predict that computation up to \gls{ode}s up $8^{th}$ order will take 44 minutes with \means{} and as much as 128 days with the original implementation.\quentintodo{Add a number saying how long it would have taken for the first iteration of optimisation as well.}
These improvement have allowed us to explore the performance of MEA in higher depth, and will hopefully contribute to make \gls{mea} realistically usable for systems with more species and reactions.

\subsection{Moment Expansion Closure}

In \gls{mea} the temporal derivative of each central moment is expressed in terms of the higher order moment. 
This behaviour essentially has no upper bound and continues to the moments of infinite order. 
Since we cannot evaluate these expressions in the limit analytically, it is necessary to ``close'' the expansion by providing a closed form for the higher order moments. This also makes the method an ``aproximation" rather than an exact method.

In the original work \cite{ale_general_2013}, higher order central moments are assumed to be equal to zero. This is very clean mathematically, but is a strong and not necessarily valid assumption. 

As an alternative, parametric probability distribution can be used to express moments of arbitrary orders. 
For instance, a multivariate normal distribution is parametrised only by means (\ie{} first order raw moment)
and a covariance matrix (\ie{} second order central moments). 
As a consequence, it is possible to express any arbitrary moment from means, variances and covariances. 
A promising area of research involves closing moment expression by parametric forms of highest order central moments instead
of assuming them to be null.
Preliminary work \citationneeded{Eszter, unpublished} suggests that using a parametric distributions for \gls{mea} \quentintodo[inline]{Finish this sentence!}
In addition, Ale \emph{et al.} predicted that higher maximal moment order would \quentintodo{are you sure? didn't Michael deny this}necessarily result in better approximation \cite{ale_general_2013}.

The dramatic improvement in performance compared to the \mat{} prototype (see \autoref{sec:optimising_mea}) has made it possible for us to explore the dynamics of both the higher-order and probabilistic moment closures and verify the claims about their performance efficiently.

\quentintodo[inline]{Consider making this figure a barchart}
\begin{figure}[t]
    \centering
    \includegraphics[width=1.0\textwidth]{../pipeline/task-output/FigureP53Summary/FigureP53Summary-pdf-7.pdf}
    \caption{\emph{Effect of different closure methods and maximal moment order on simulation accuracy}. The \pft system was modelled using \gls{mea} with five types of closure and for maximal moment order up to seven.
Resulting trajectories were all compared to an average of 5000 \gls{gssa} simulations using sum of square distance (a).
Distance is in log scale. Missing values indicate solver failure for that particular set of parameters.}
    \label{fig:max_order_and_closure_on_distance_summary}
\end{figure}


Figure~\ref{fig:max_order_and_closure_on_distance_summary} summarises the effect of increasing maximal moment order and different type of closures.
The \pft{} system, with parameters from \cite{ale_general_2013}, was investigated.
As mentioned earlier it is expected that the approximation accuracy increases as the maximal moment order does, \ie{} distance to \gls{gssa} obtained means decreases as maximal moment order increases. 
This trend is true for for normal and scalar closures, up to maximal moment order six. 

This trend is not true for the seventh order moment, though.
\quentintodo[inline]{Todo double check the labels in the legend, was it really normal that was worse and scalar that failed?}
As seen in the figure, the closure using the normal distribution had performed worse than the sixth order closure when maximum order was set to seven.
We could not obtain the result for the seventh order closure using the standard scalar closure, as the ODE solver has failed simulating the problem, which usually indicates a stiff \gls{ode} problem. 
The generated trajectories are shown in the \autoref{fig:max_order_and_closure_on_distance_trajectories}. One can see the trajectory generated from the normal distribution closure mismatch the trajectory obtained from the \gls{gssa} simulations by quite a wide margin (\emph{purple line, subplot on the right-hand-side}). 

It is hard to determine the reason for this behaviour, it might be a limitation of the approximation method, or a limitation of the \gls{ode} solver available. It is hard to know which explanation is more likely as, we cannot test the two hypotheses separately. All we can say is that have been able to observe a similar behaviour with all of the solvers tested, which is further explored in \sauliustodo{link to my section where we study the phenomenon on larger scale with different solvers}.

\quentintodo{I cannot see that to be true, from the figure}
Note that for even maximal moment orders, normal and scalar are rigorously equal.
\quentintodo{Not intuitive to me, explain}This is expected since normal distribution is symmetrical (\emph{i.e.} odd central moments are always zero).

\begin{figure}
    \centering
    \includegraphics[width=1.0\textwidth]{../pipeline/task-output/FigureP53Simple/FigureP53Simple-pdf-7.pdf}
%~
    \caption{\emph{Complete trajectories of a single species (\pft) for max order three and seven are shown} Black lines indicate the average of \gls{gssa} simulations. Missing lines, compared to the legend in \autoref{fig:max_order_and_closure_on_distance_summary} indicate solver failure.}
    \label{fig:max_order_and_closure_on_distance_trajectories}
\end{figure}

Log-normal closure method is also displaying a similar behaviour with regards to the approximation of ground truth trajectories, though less extreme. 
The ground truth trajectory seems to be well approximated when the maximal moment order is three, but the approximation gets less and less accurate for higher maximal order moments.
A deeper look at the trajectories indicate that, in this latter case,
oscillations are damped too quickly, as opposed to the behaviour seen using normal closure, where the oscillation amplitude increases (see \autoref{fig:max_order_and_closure_on_distance_trajectories}, red line).

\quentintodo{what?} Interestingly for even maximum moment order log-normal closures generated \gls{ode}s which,
despite our efforts, could not be numerically solved.

Finally, it seems that the results obtained from multivariate distribution closures and  univariate distribution closures, which do not model the covariance terms, are the same for this particular system. 
This is not true for all of the systems. For instance, we have observed that in the \emph{hes1}, it is advantageous to model covariance (data not show\quentintodo{show data, otherwise our point is weak}).

The results in this section show that, while it might be convenient to believe that using higher order moments would provide better approximations, the results indicate that this is not necessarily the case as there seems to be the \emph{sweet spot} parameter set that gives the best results. Unfortunately, this makes it difficult to \emph{a priori} define which closure and maximal moment order should be used for a given system. The section \sauliustodo{link to the section} explores this phenomenon on wider scale of parameters.

\subsection{Parameter Inference using \acrlong{mea}}
Parameter inference procedure aims to obtain the correct parameter values for the system by exploring the parameter space and comparing the simulation trajectories with the experimental data. 

In order to study the performance of the parameter inference procedure using \acrlong{mea}, we took the average trajectory obtained from the  $5000$ \gls{gssa} simulations of the \pft{} with a certain parameter set and used it as the observed dataset we want to infer the parameters from. Essentially, the inference procedure is expected to return the said parameter set back to us.

In order to determine the base case performance, we performed the parameter inference using the \pft{} model expressed only in terms of the first order moments. This approximation is bound to be very inaccurate for the particular system, as higher order moments are necessary to capture the damped oscillations present in the means of \gls{gssa} simulations\cite{ale_general_2013}\sisitodo{figure is needed}.

We have decided started parameter inference procedure at the correct parameter values, expecting to see an immediate result and no movement in the parameter space. To much of our surprise, however, when the inference is allowed to navigate the complete parameter space, not only did the procedure the procedure did move around, but it was able to find a set of parameters, that when expressed in terms of first order moments, are able to match the observed trajectories completely.

The set of parameters obtained by the inference were very different from the actual SSA simulations, but the approximated trajectories were identical. This strongly questions the suitability of parameter inference procedures in obtaining the correct parameter sets.

\sisitodo[inline]{Need figure here showing that they are really close}
\sisitodo[inline]{Why don't we compare the SSA trajectories with these parameters somewhere as well, if there is time?}

In order to gain some insight into what is happening, we first attempted to restrict ourselves to easier to comprehend two-dimensional space.
To do this, we restricted our parameter inference procedure to only move through pair of parameters only, while all other parameters are fixed (to their true values). 
We performed this for all combinations of two parameters and checked if we can reproduce the curious behaviour.

Interestingly, we were able to observe the same behaviour for the parameter pairs ....
\sisitodo[inline]{list the parameter pairs that work, add figure for $c_2$, $c_6$.}
We have chosen the pair of parameters that allowed the inference procedure to converge to a trajectory with minimal distance to the stochastic average -- $c_2$ and $c_6$ for further investigations.

\sisitodo[inline]{look up autoref function, and refer to correct figure, it is not Figure 5 any more, as it was before}
 
\sisitodo[inline]{See the commented text below this todo -- instead of it please talk a bit more about how the landscape looks for the max\_order = 1. Then go on and talk about higher-order moments, explicitly mentioning that you were able to find close-enough trajectories for all of them, however, neither of them were accurate parameter wise}
%If the inference method was correct, the starting point in the distance landscape would already have a low distance, and the end point should overlap with the starting point, i.e. the true values of $c_2$ and $c_6$.

In conclusion, the data makes us question the validity of parameter inference approaches using \gls{mea} approximation.
As seen in \sisitodo{remind which figures} distance landscape shows that the starting point (which is the correct parameter set) can be distant from the minima. Similarly, the distance procedure, if uncontrolled, is able to find itself exploring the values for some parameters (i.e. $c_6$) that are more than 10 times larger than the true value, which not be very meaningful biologically.\sisitodo{great point, but also needs coverage above}
\sisitodo{not clear what you meant here}Finally, the trajectories for all the species in the p53 model sometime demonstrate misfit between the optimal trajectory obtained from inference and the "observed trajectory" from \gls{gssa}. 

\begin{figure}
\centering
    \begin{subfigure}[t]{0.2\textwidth}
    \includegraphics[width=\textwidth]{{../pipeline/task-output/SampleMultidimensionInferenceFigure/SampleMultidimensionInferenceFigure-pdf-1-scalar-True-90.0_0.002_1.704_1.1_0.93_0.96_0.7822-ode15s--90.0_0.002_1.704_1.1_0.93_0.96_0.7822-sum_of_squares-5000}.pdf}
    \end{subfigure}
    \begin{subfigure}[t]{0.2\textwidth}
    \includegraphics[width=\width]{{../pipeline/task-output/SampleMultidimensionInferenceFigure/SampleMultidimensionInferenceFigure-pdf-2-scalar-True-90.0_0.002_1.704_1.1_0.93_0.96_0.7822-ode15s--90.0_0.002_1.704_1.1_0.93_0.96_0.7822-sum_of_squares-5000}.pdf}
    \end{subfigure}
    \begin{subfigure}[b]{0.2\textwidth}
    \includegraphics[width=\textwidth]{{../pipeline/task-output/SampleMultidimensionInferenceFigure/SampleMultidimensionInferenceFigure-pdf-3-scalar-True-90.0_0.002_1.704_1.1_0.93_0.96_0.7822-ode15s--90.0_0.002_1.704_1.1_0.93_0.96_0.7822-sum_of_squares-5000}.pdf}
    \end{subfigure}
    \begin{subfigure}[b]{0.2\textwidth}
    \includegraphics[width=\textwidth]{{../pipeline/task-output/SampleMultidimensionInferenceFigure/SampleMultidimensionInferenceFigure-pdf-4-scalar-True-90.0_0.002_1.704_1.1_0.93_0.96_0.7822-ode15s--90.0_0.002_1.704_1.1_0.93_0.96_0.7822-sum_of_squares-5000}.pdf}
    \end{subfigure}
    \begin{subfigure}[b]{0.2\textwidth}
    \includegraphics[width=\textwidth]{{../pipeline/task-output/SampleMultidimensionInferenceFigure/SampleMultidimensionInferenceFigure-pdf-5-scalar-True-90.0_0.002_1.704_1.1_0.93_0.96_0.7822-ode15s--90.0_0.002_1.704_1.1_0.93_0.96_0.7822-sum_of_squares-5000}.pdf}
    \end{subfigure}
    
\caption{\emph{Distance landscape at different maximal orders for p53 model.} In the landscape, the warmer the colour, the more distant the inferred trajectories are from the \gls{gssa} trajectories. 
The \gls{gssa} trajectories are generated using the new parameter values labelled as \emph{start}. Among seven parameters, only the values for $c_2$ and $c_6$ are inferred, with starting values inferred from inference using the true values.} 
\label{fig:parameter_inference_landscape}
\end{figure}

\begin{figure}
\centering
    \begin{subfigure}[b]{0.6\textwidth}
    \includegraphics[width=\textwidth]{{../pipeline/task-output/FigureInferenceStartEndSSA/FigureInferenceStartEndSSA-1-scalar-c2-1.7040-c6-0.7822}.pdf}
    \end{subfigure}
    \begin{subfigure}[b]{0.6\textwidth}
    \includegraphics[width=\textwidth]{{../pipeline/task-output/FigureInferenceStartEndSSA/FigureInferenceStartEndSSA-2-scalar-c2-1.7040-c6-0.7822}.pdf}
    \end{subfigure}
    \begin{subfigure}[b]{0.6\textwidth}
    \includegraphics[width=\textwidth]{{../pipeline/task-output/FigureInferenceStartEndSSA/FigureInferenceStartEndSSA-3-scalar-c2-1.7040-c6-0.7822}.pdf}
    \end{subfigure}
    \begin{subfigure}[b]{0.6\textwidth}
    \includegraphics[width=\textwidth]{{../pipeline/task-output/FigureInferenceStartEndSSA/FigureInferenceStartEndSSA-4-scalar-c2-1.7040-c6-0.7822}.pdf}
    \end{subfigure}
    \begin{subfigure}[b]{0.6\textwidth}
    \includegraphics[width=\textwidth]{{../pipeline/task-output/FigureInferenceStartEndSSA/FigureInferenceStartEndSSA-5-scalar-c2-1.7040-c6-0.7822}.pdf}
    \end{subfigure}
\caption{\emph{Trajectories for each species in p53 model using different maximal orders.} 
Three trajectories are shown for each species. 
The starting trajectories are simulated using the starting values are indicated above the trajectories, with $c_2$ and $c_6$ inferred using \emph{sum of squares} distance method. 
Both the optimal and the \gls{gssa} trajectories are generated based on the end point in correspondent distance landscape in Figure 5.} 
\label{fig:parameter_inference_trajectories}
\end{figure}

\newpage{}
\section{Conclusion} \label{sec:conclus}

The goal of this project was to develop an efficient \py{} implementation of \acrlong{mea}.
We aimed at providing a tool that would be powerful whilst simple to maintain and extend.
After thoroughly considering the requirements and potential impact, we decided to write a \py{} package as opposed to a command line application.

Our work built on top of the available \mat{} code as well as last year's MSc students' project, which we eventually completely restructured.
We used state-of-the-art software development standards, such as version control, continuous integration and unittests to manage our project and speed-up development.
In order to improve usability and maintainability, we provide an exhaustive series of tutorials and a detailed package documentation.

In addition, we implemented moment expansion closure with three parametric distributions (in their univariate and multivariate form).
In order to maximise impact for adoption of \means, we also implemented support for the SBML standard.
For completeness, and to allow comparison between different algorithms, \gls{gssa} and \gls{lna} were also implemented.
A considerable effort was made to improve the performance of \gls{mea},  in both symbolic computations and numerical evaluation.
We also provide a comprehensive tutorial showing detailed steps of executing each function of the package.

These improvements allowed us to investigate the effect of maximal moment order and closure distribution on \gls{mea} accuracy.
In contrast with the original publication, we find that the approximation does not necessarily improve with maximal moment order.
In addition, we discovered that according to the region of the parameter space, the best approximation (maximal order and closure distribution) could be very different.
We also realised that for large regions of the parameter space, all numerical solvers failed to simulate trajectories.
Finally, we pointed out that parameter inference based on \gls{mea} approximation cannot produce the expected parameter values, even though the inferred species trajectories can be a perfect match to the ``experimental'' data. Using higher order for \gls{mea} may improve the accuracy of parameter inference slightly, however, the true parameter values cannot be recovered even using maximal order of 6.
Reporting these unforeseen limitations will hopefully help researchers to improve simulation and inference using \gls{mea}.

Altogether, our implementation should facilitate further research and improve adoption of \acrlong{mea} by the community.
We believe \means{} could be a valuable assess for researchers to model non-linear kinetic system.
 
%We intend to release \means as open source software available 


\begin{verbatim}
* (Un)Availability of our package
\end{verbatim}

\todo[inline]{Once we are all good, we need to zip the task-output data and make it share it somewhere as well as whole report depends on it}

\newpage{}
\bibliography{report.bib}{}
\bibliographystyle{ieeetr}

\newpage{}
\begin{appendices}
\section{Documentation}
\label{sec:documentation}
\sauliustodo[inline]{add docs here}
\end{appendices}
   
\end{document}


 
