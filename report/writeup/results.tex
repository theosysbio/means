\section{Results and Discussion} \label{results}


\subsection{Performance}\label{performance}
The temporal dynamic of a molecular system can be described by the \gls{cme}.
%~ \cite{}
\gls{gssa}\cite{gillespie_general_1976} can produce individual realisations of the \gls{cme}.
In order to obtain accurate estimates of the average dynamic within a population of cell, it it however necessary to perform multiple (often more than $10^4$) simulations.
Despite recent effort \cite{niemi_efficient_2011,dittamo_optimized_2009,komarov_accelerating_2012} to provide fast implementation of this algorithm, computation remains extremely expensive.
This is critical when performing, for instance, parameter inference. 
This limitation led to the development of approximations such as \gls{lna}\cite{komorowski_bayesian_2009} and \gls{mea}\cite{ale_general_2013} which can perform in a more reasonable time.

Since the only advantage of \gls{mea} over \gls{gssa} is computational speed, it was paramount to provide an efficient an implementation of \gls{mea}.
In this section, we show that symbolic computations can be limiting for \gls{mea}. 
Then, we explain how we have optimised them, and finally increase performance by several order of magnitudes over to the original \mat{} implementation.
In addition, ...\\
TODO @saul NUMERICAL EVAL/solvers.


\subsubsection{Symbolic expressions}


