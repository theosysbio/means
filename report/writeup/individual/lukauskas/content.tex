This document summarises the major contributions of Saulius Lukauskas
to the development of \means{} package.
This document has been written as a succinct supplement to the group report, and therefore it contains numerous back-references to sections of the larger document.

\section{Introduction}

\section{Contributions to Software Development Process}

In the modern day, the computers are becoming the driving force of innovation and progress. 
The whole world is facing the need to either adjust to this computerisation or risk to be left behind by others.
Naturally, the need to use and computer software is becoming increasingly important for scientific communities in every major field of science and has already reached the extent that some knowledge of computer programming is becoming a necessary skill.

While it has been accepted that knowledge of computer programming is necessary, the need to study software-engineering is still being debated. 
The common argument for the lack of need to study software engineering, claims that people often care only about the end result, and not how it was obtained or how easy-to-read code is. 
I find this claim ironic as the same argument is actually the main reason why the software-engineering methodologies emerged in the first place. 
For instance, two of the main principles of the \emph{agile} methodology\cite{_manifesto_????} claim treat the ``working software as the primary measure of success" and claim that ``simplicity -- the art of maximising of work \emph{not} done -- is essential"\cite{paulk_agile_2002}, clearly in par with the same argument described earlier. 
In fact \emph{agile} methodology is viewed as an alternative to the approaches that aim to create this idealistic, properly structured and well-documented software, such as the traditional \emph{waterfall} model.

It was my personal goal in this project to share my experience in software engineering in order to incorporate the \emph{agile} methodology into our development cycle. 
I aimed to both introduce my peers to the best practices from the industry and to immediately validate them by showing how these practices allow reaching the desired goal faster.

\subsection{Automated Testing}

\subsection{Code Structure}
\subsection{Continuous Integration}
\subsection{Implementation of Simulation and Inference Routines}
\section{Investigation of MEA and Solver Performance}
\section{Pipeline Support}
\section{Conclusions}

