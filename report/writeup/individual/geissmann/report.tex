\documentclass[11pt,a4paper]{article}
\setcounter{secnumdepth}{6}

\usepackage{standalone}
\usepackage{geometry} % Used to adjust the document margins
\usepackage{graphicx}
\usepackage[font=footnotesize]{caption}
\usepackage[noadjust]{cite}
\usepackage[toc,page]{appendix}
\usepackage{hyperref}
\usepackage{fullpage}
\usepackage{amsmath}
\usepackage{sidecap}
\usepackage{caption}
\usepackage{subcaption}
\usepackage{minted}
\usepackage{framed}
% \usepackage[colorinlistoftodos]{todonotes}
\usepackage[disable]{todonotes}
\usepackage{csquotes}
\usepackage{parskip} % Spaces between paragraphs

\usepackage{pgffor} % foreach loops!
\usepackage[acronym]{glossaries}

\usepackage{setspace}
\doublespacing

% refering back to group report
\usepackage{xr}
\externaldocument[GR:]{../../report}

% add binding margins
\geometry{bindingoffset=1cm}

\makeglossaries
\renewcommand*\abstractname{Summary}

% OUR COMMANDS
%~ 
%~ \newenvironment{pythoncode}%
%~ {\begin{framed}
%~ \begin{minted}{python}}%
%~ {\end{minted}
%~ \end{framed}}%

% http://www.tex.ac.uk/cgi-bin/texfaq2html?label=altabcr
\setcounter{MaxMatrixCols}{50}

% package name:
\newcommand{\means}{\texttt{MEANS}}
\newcommand{\pft}{\textit{p53}}
\newcommand{\py}{\texttt{python}}
\newcommand{\sympy}{\texttt{sympy}}
\newcommand{\plt}{\texttt{matplotlib}}
\newcommand{\mat}{\texttt{MATLAB}}
\newcommand{\eg}{\emph{e.g.}}
\newcommand{\ie}{\emph{i.e.}}

\newcommand{\sauliustodo}[2][]{\todo[color=cyan, #1]{\textbf{SL:} #2}}
\newcommand{\sisitodo}[2][]{\todo[color=yellow, #1]{\textbf{SF:} #2}}
\newcommand{\quentintodo}[2][]{\todo[color=red, #1]{\textbf{QG:} #2}}
\newcommand{\citationneeded}[2][]{\todo[color=brown, fancyline, #1]{\textbf{Citation Needed:} #2}}
\newcommand{\contrib}{\emph}
\begin{document}

\listoftodos
\newpage

\title{MEANS: a new python package for Moment Expansion Approximation, Inference and Simulation}
\author{Individual report: Quentin Geissmann \\
\\	
\\
\\
\\
Supervised by Ann Babtie, Paul Kirk, Eszter Lakatos and Michael Stumpf\\
\\
\\
Theoretical Systems Biology Group,\\
Imperial College London
}
\date{\today}

\clearpage\maketitle
\thispagestyle{empty}
\newpage{}

\pagenumbering{roman}

\pagebreak % move TOC to a different page from abstract
\begin{table}
\begin{center}   
    \begin{tabular}{ | l  | r|}
    \hline
     \multicolumn{2}{|c|}{\bf{Word count}}\\
    \hline
    \hline
    Main text & 2539\\
    \hline
    Figures & 363\\
    \hline
    Headers & 31\\
    \hline
    \end{tabular}
\end{center}
\end{table}

\tableofcontents

% our acronyms
\newacronym{ode}{ODE}{Ordinary Differential Equation}
\newacronym{mea}{MEA}{Moment Expansion Approximation}
\newacronym{lna}{LNA}{Linear Noise Approximation}
\newacronym{ssa}{SSA}{Gillespie Stochastic Simulation Algorithm}
\newacronym{cme}{CME}{Chemical Master Equation}
\newacronym{abc}{ABC}{Approximate Bayesian Computation}
\newacronym{sbml}{SBML}{Systems Biology Markup Language}
\newacronym{pypi}{PyPI}{the Python Package Index}

\newglossaryentry{maxord}
{
  name=maximal order,
  description={Maximal moment expansion order. Max order always (regardless to the closure method) refers to the highest order of 
  moments present in an ODE system resulting from Moment Expansion Approximation. 
  In other words, moment expansion is closed with moments of maximal order plus one.
   },
  sort=maximal order
}
\newpage{}
\printglossaries


\section{Introduction}
\section{Contributions to Software Development Process}
\subsection{Unit Tests}
\subsection{Code Structure}
\subsection{Continuous Integration}
\subsection{Implementation of Simulation and Inference Routines}
\section{Investigation of MEA and Solver Performance}
\section{Pipeline Support}


\newpage{}
\bibliography{../../report.bib}{}
\bibliographystyle{ieeetr}

\end{document}
